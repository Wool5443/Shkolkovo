\documentclass[14pt,a4paper]{article}
\usepackage[utf8]{inputenc}
\usepackage[russianb]{babel}
\usepackage[left=1.5cm,right=1.5cm,top=2cm,bottom=2.5cm]{geometry}
\usepackage{setspace}
\usepackage{indentfirst}
\usepackage{amssymb}
\usepackage{amsmath}

\usepackage{array}
\usepackage[pdftex]{graphicx}
\usepackage{comment}
\usepackage[table,xcdraw]{xcolor}


\usepackage{verbatim}


\graphicspath{{images/}}
\renewcommand{\baselinestretch}{1.3}

\begin{document}

Необходимо пройтись по всем парам соседних чисел и найти такие, среди
которые хотя бы одно число больше максимального числа в файле,
кратного 8. Для этого сначала найдём это самое максимальное число.
После чего можно начать перебирать пары чисел.

\begin{verbatim}
# Откроем файл и считаем числа в список
file = open("17__1z92t.txt")
numbers = [int(i) for i in file]

# Создадим переменные для счётчика и минимальной суммы
count = 0
max_sum = 10005000

# Переменная для максимального числа, кратного 8
mx = 0
# Пройдёмся по всем числам
for i in range(len(numbers)):
    # Обновляем максимальное число
    if (numbers[i] > mx) and (numbers[i] % 8 == 0):
        mx = numbers[i]

# Пройдёмся по парам
for i in range(len(numbers) - 1):
    # Проверяем, что хотя бы одно число в паре больше mx
    if (numbers[i] > mx) or (numbers[i + 1] > mx):
        # Увеличиваем счётчик
        count += 1
        # Обновляем минимальную сумму
        if numbers[i] + numbers[i + 1] < max_sum:
            max_sum = numbers[i] + numbers[i + 1]

# Выведем ответ
print(count, max_sum)
\end{verbatim}

\end{document}
