\documentclass[14pt,a4paper]{article}
\usepackage[utf8]{inputenc}
\usepackage[russianb]{babel}
\usepackage[left=1.5cm,right=1.5cm,top=2cm,bottom=2.5cm]{geometry}
\usepackage{setspace}
\usepackage{indentfirst}
\usepackage{amssymb}
\usepackage{amsmath}

\usepackage{array}
\usepackage[pdftex]{graphicx}
\usepackage{comment}
\usepackage[table,xcdraw]{xcolor}


\usepackage{verbatim}


\graphicspath{{images/}}
\renewcommand{\baselinestretch}{1.3}

\begin{document}

В файле 17.txt содержится последовательность целых чисел. Элементы
последовательности могут принимать целые значения от 0 до 10 000
включительно. Определите количество пар, в которых хотя бы один
элемент больше, чем среднее арифметическое всех чисел в файле,
кратных 4. В ответе запишите два числа: сначала количество найденных
пар, а затем -- максимальную сумму элементов таких пар. В данной
задаче под парой подразумевается два идущих подряд элемента
последовательности.

\end{document}
