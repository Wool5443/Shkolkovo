\documentclass[14pt,a4paper]{article}
\usepackage[utf8]{inputenc}
\usepackage[russianb]{babel}
\usepackage[left=1.5cm,right=1.5cm,top=2cm,bottom=2.5cm]{geometry}
\usepackage{setspace}
\usepackage{indentfirst}
\usepackage{amssymb}
\usepackage{amsmath}

\usepackage{array}
\usepackage[pdftex]{graphicx}
\usepackage{comment}
\usepackage[table,xcdraw]{xcolor}


\usepackage{verbatim}


\graphicspath{{images/}}
\renewcommand{\baselinestretch}{1.3}


\begin{document}

Нужные нам тройки должны удовлетворять двум условиям:
\begin{enumerate}
    \item Хотя бы одно число оканчивается на 7. Для этого можно проверить остаток от деления на 10, если он равен 7, то число подходит. \\
    \item Сумма всех трёх чисел кратна 6. \\
\end{enumerate}
Необходимо посчитать их количество, а также найти минимальную сумму чисел среди таких троек.

\begin{verbatim}

# Откроем файл и считаем числа в список a
f = open("17__1z92x.txt")
a = [int(i) for i in f]

# Заведём переменные для счётчика и минимальной суммы
c = 0
mn = 10005000

# Переберём последовательные тройки
for i in range(len(a) - 2):
    # Проверим, что хотя бы одно число в тройке оканчивается на 7
    if ((a[i] % 10 == 7) or (a[i + 1] % 10 == 7) or (a[i + 2] % 10 == 7)):
        # Проверим, что сумма тройки кратна 6
        if (a[i] + a[i + 1] + a[i + 2]) % 6 == 0:
            # Тройка подошла -- увеличиваем счётчик
            c += 1
            # Обновляем минимальную сумму
            if a[i] + a[i + 1] + a[i + 2] < mn:
                mn = a[i] + a[i + 1] + a[i + 2]

# Выводим ответ
print(c, mn)


\end{verbatim}
\end{document}
