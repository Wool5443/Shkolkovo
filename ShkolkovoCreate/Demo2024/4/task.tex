\documentclass[14pt,a4paper]{article}
\usepackage[utf8]{inputenc}
\usepackage[russianb]{babel}
\usepackage[left=1.5cm,right=1.5cm,top=2cm,bottom=2.5cm]{geometry}
\usepackage{setspace}
\usepackage{indentfirst}
\usepackage{amssymb}
\usepackage{amsmath}

\usepackage{array}
\usepackage[pdftex]{graphicx}
\usepackage{comment}
\usepackage[table,xcdraw]{xcolor}


\usepackage{verbatim}


\graphicspath{{images/}}
\renewcommand{\baselinestretch}{1.3}

\begin{document}

По каналу связи передаются сообщения, содержащие только восемь букв:
А, Б, В, Г, Д, Е, Ж и З. Для передачи используется двоичный код,
удовлетворяющий условию Фано.

Кодовые слова для некоторых букв известны:

\begin{center}
	\begin{tabular}{|c|c|}
		\hline
		А        & 000                    \\ \hline
		Б        & 001                    \\ \hline
		В        & 0101                   \\ \hline
		Г        & 0100                   \\ \hline
		Д        & 011                    \\ \hline
		Е        & 101                    \\ \hline
	\end{tabular}
\end{center}

Какое \textbf{наименьшее} количество двоичных знаков потребуется для
кодирования двух оставшихся букв?

В ответе запишите суммарную длину кодовых слов для букв: Ж, З.

\textit{Примечание}. Условие Фано означает, что никакое кодовое слово
не является началом другого кодового слова. Это обеспечивает
возможность однозначной расшифровки закодированных сообщений.

\end{document}
