\documentclass[14pt,a4paper]{article}
\usepackage[utf8]{inputenc}
\usepackage[russianb]{babel}
\usepackage[left=1.5cm,right=1.5cm,top=2cm,bottom=2.5cm]{geometry}
\usepackage{setspace}
\usepackage{indentfirst}
\usepackage{amssymb}
\usepackage{amsmath}
\usepackage{bm}

\usepackage{array}
\usepackage[pdftex]{graphicx}
\usepackage{comment}
\usepackage[table,xcdraw]{xcolor}


\usepackage{verbatim}


\graphicspath{{images/}}
\renewcommand{\baselinestretch}{1.3}

\begin{document}

Исполнитель Черепаха действует на плоскости с декартовой системой
координат. В начальный момент Черепаха находится в начале координат,
её голова направлена вдоль положительного направления оси ординат,
хвост опущен. При опущенном хвосте Черепаха оставляет на поле след в
виде линии. В каждый конкретный момент известно положение исполнителя
и направление его движения. У исполнителя существует две команды:
\textbf{Вперёд} $\bm{n}$ (где $n$ -- целое число), вызывающая
передвижение Черепахи на $n$ единиц в том направлении, куда указывает
её голова, и \textbf{Направо} $\bm{m}$ (где $m$ -- целое число),
вызывающая изменение направления движения на $m$ градусов по часовой
стрелке.

Запись \textbf{Повтори} $\bm{k}$ \textbf{[Команда1 Команда2 …
КомандаS]} означает, что последовательность из $S$ команд повторится
$k$ раз.

Черепахе был дан для исполнения следующий алгоритм:

\textbf{Повтори 7 [Вперёд 10 Направо 120].}

Определите, сколько точек с целочисленными координатами будут
находиться внутри области, которая ограничена линией, заданной этим
алгоритмом. Точки на линии учитывать не следует.

\end{document}
