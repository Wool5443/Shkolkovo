\documentclass[14pt,a4paper]{article}
\usepackage[utf8]{inputenc}
\usepackage[russianb]{babel}
\usepackage[left=1.5cm,right=1.5cm,top=2cm,bottom=2.5cm]{geometry}
\usepackage{setspace}
\usepackage{indentfirst}
\usepackage{amssymb}
\usepackage{amsmath}
\usepackage{bm}

\usepackage{array}
\usepackage[pdftex]{graphicx}
\usepackage{comment}
\usepackage[table,xcdraw]{xcolor}


\usepackage{verbatim}


\graphicspath{{images/}}
\renewcommand{\baselinestretch}{1.3}

\begin{document}

Исполнитель Черепаха действует на плоскости с декартовой системой
координат. В начальный момент Черепаха находится в начале координат,
её голова направлена вдоль положительного направления оси ординат,
хвост опущен. При опущенном хвосте Черепаха оставляет на поле след в
виде линии. В каждый конкретный момент известно положение исполнителя
и направление его движения. У исполнителя существует 6 команд:
\textbf{Поднять хвост}, означающая переход к перемещению без
рисования; \textbf{Опустить хвост}, означающая переход в режим
рисования; \textbf{Вперёд} $\bm{n}$ (где $n$ – целое число),
вызывающая передвижение Черепахи на $n$ единиц в том направлении,
куда указывает её голова; \textbf{Назад} $\bm{n}$ (где $n$ – целое
число), вызывающая передвижение в противоположном голове направлении;
\textbf{Направо} $\bm{m}$ (где $m$ – целое число), вызывающая
изменение направления движения на $m$ градусов по часовой стрелке,
\textbf{Налево} $\bm{m}$ (где $m$ – целое число), вызывающая
изменение направления движения на $m$ градусов против часовой стрелки.

Запись \textbf{Повтори} $\bm{k}$ \textbf{[Команда1 Команда2 …
КомандаS]} означает, что последовательность из S команд повторится k
раз.

Черепахе был дан для исполнения следующий алгоритм:

\textbf{Повтори 2 [Вперёд 8 Направо 90 Вперёд 18 Направо 90]}

\textbf{Поднять хвост}

\textbf{Вперёд 4 Направо 90 Вперёд 10 Налево 90}

\textbf{Опустить хвост}

\textbf{Повтори 2 [Вперёд 17 Направо 90 Вперёд 7 Направо 90]}

Определите, сколько точек с целочисленными координатами будут
находиться внутри объединения фигур, ограниченного заданными
алгоритмом линиями, включая точки на линиях.

\end{document}
