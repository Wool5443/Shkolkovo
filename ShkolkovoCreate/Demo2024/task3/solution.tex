\documentclass[14pt,a4paper]{article}
\usepackage[utf8]{inputenc}
\usepackage[russianb]{babel}
\usepackage[left=1.5cm,right=1.5cm,top=2cm,bottom=2.5cm]{geometry}
\usepackage{setspace}
\usepackage{indentfirst}
\usepackage{amssymb}
\usepackage{amsmath}

\usepackage{array}
\usepackage[pdftex]{graphicx}
\usepackage{comment}
\usepackage[table,xcdraw]{xcolor}


\usepackage{verbatim}


\graphicspath{{images/}}
\renewcommand{\baselinestretch}{1.3}

\begin{document}

Для каждой операции нам нужно знать адрес, наименование товара и его вес.
Поместим адрес в столбец G, для этого в G2 запишем формулу и растянем:
\begin{center}
    =ВПР(C2;\$Магазин.A:C;3;0)
\end{center}
Наименование товара поместим в столбец H, в H2 запишем и растянем:
\begin{center}
    =ВПР(D2;\$Товар.A:F;3;0)
\end{center}
Вес одной единицы товара поместим в столбец I, в I2 запишем и растянем:
\begin{center}
    =ВПР(D2;\$Товар.A:F;5;0)
\end{center}
Теперь посчитаем общий вес операции, записав в J2 следующую формулу и растянув:
\begin{center}
    =I2*E2
\end{center}

Далее нужно применить фильтры. Сначала поставим фильтр на дату от 2 до 10 августа включительно, затем поставим тип операции "Поступление". Выберем адреса на проспекте Революции, в товарах выберем все виды зефира. Скопируем полученную таблицу на новый лист и посчитаем ответ. В K2 запишем:
\begin{center}
    =СУММ(J:J)/1000
\end{center}
Получим ответ 3570 кг.

\end{document}
