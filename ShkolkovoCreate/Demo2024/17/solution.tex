\documentclass[14pt,a4paper]{article}
\usepackage[utf8]{inputenc}
\usepackage[russianb]{babel}
\usepackage[left=1.5cm,right=1.5cm,top=2cm,bottom=2.5cm]{geometry}
\usepackage{setspace}
\usepackage{indentfirst}
\usepackage{amssymb}
\usepackage{amsmath}
\usepackage{bm}

\usepackage{array}
\usepackage[pdftex]{graphicx}
\usepackage{comment}
\usepackage[table,xcdraw]{xcolor}


\usepackage{verbatim}


\graphicspath{{images/}}
\renewcommand{\baselinestretch}{1.3}

\begin{document}

Для решения напишем программу. Откроем файл и считаем из него числа.
Затем нужно найти максимальное число в файле, оканчивающееся на \textbf{13}.
Чтобы посмотреть, на какие 2 цифры оканичвается число, можно
посмотреть его остаток от деления на 100. Введём счётчик подходящих
троек и переменную для максимальной суммы.

Далее нужно пройтись по тройкам, посчитать у каждой сумму и сколько в
ней трёхзначных чисел. Подходящими парами являются пары, в которых
\textbf{2 трёхзначных числа}, а сумма \textbf{не больше}
максимального числа \textbf{в файле}, кратного \textbf{13}.

\begin{verbatim}
# Открываем файл
file = open("17_2024.txt")

# Считываем числа из файла
numbers = [int(s) for s in file]

# Находим максимальное число, оканчивающееся на 13
max_div13 = max(x for x in numbers if x % 100 == 13)

# Вводим счётчик подходящик троек и переменную для максимальной суммы
count = 0
max_sum = 0

# Проходимся по тройкам
for i in range(len(numbers) - 2):
    # Срезом берём текущую тройку
    triplet = numbers[i:i+3]
    # Находим сумму тройки
    triplet_sum = sum(triplet)
    # Считаем, сколько трёхзначных чисел в тройке
    count_three_digits = sum(100 <= x <= 999 for x in triplet)

    # Проверяем, что два трёхзначных чисел
    # и сумма тройки не больше максимального числа, кратного 13
    if count_three_digits == 2 and triplet_sum <= max_div13:
        # Увеличиваем счётчик
        count += 1
        # Обновляем максимальную сумму
        max_sum = max(max_sum, triplet_sum)

# Выводим ответ
print(count, max_sum)  # 959 97471
\end{verbatim}

\end{document}
