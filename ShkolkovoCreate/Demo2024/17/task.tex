\documentclass[14pt,a4paper]{article}
\usepackage[utf8]{inputenc}
\usepackage[russianb]{babel}
\usepackage[left=1.5cm,right=1.5cm,top=2cm,bottom=2.5cm]{geometry}
\usepackage{setspace}
\usepackage{indentfirst}
\usepackage{amssymb}
\usepackage{amsmath}
\usepackage{bm}

\usepackage{array}
\usepackage[pdftex]{graphicx}
\usepackage{comment}
\usepackage[table,xcdraw]{xcolor}


\usepackage{verbatim}


\graphicspath{{images/}}
\renewcommand{\baselinestretch}{1.3}

\begin{document}

В файле содержится последовательность натуральных чисел, каждое из
которых не превышает 100 000. Определите количество троек элементов
последовательности, в которых ровно два из трёх элементов являются
трёхзначными числами, а сумма элементов тройки не больше
максимального элемента последовательности, оканчивающегося на 13.
Гарантируется, что в последовательности есть хотя бы одно число,
оканчивающееся на 13. В ответе запишите количество найденных троек
чисел, затем максимальную из сумм элементов таких троек. В данной
задаче под тройкой подразумевается три идущих подряд элемента
последовательности.

\end{document}
