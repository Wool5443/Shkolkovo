\documentclass[14pt,a4paper]{article}
\usepackage[utf8]{inputenc}
\usepackage[russianb]{babel}
\usepackage[left=1.5cm,right=1.5cm,top=2cm,bottom=2.5cm]{geometry}
\usepackage{setspace}
\usepackage{indentfirst}
\usepackage{amssymb}
\usepackage{amsmath}

\usepackage{array}
\usepackage[pdftex]{graphicx}
\usepackage{comment}
\usepackage[table,xcdraw]{xcolor}


\usepackage{verbatim}


\graphicspath{{images/}}
\renewcommand{\baselinestretch}{1.3}

\begin{document}

По каналу связи передаются шифрованные сообщения, содержащие только
десять букв: $A$, $B$, $C$, $D$, $E$, $F$, $S$, $X$, $Y$, $Z$; для
передачи используется неравномерный двоичный код. Для кодирования
букв используются кодовые слова.

\begin{center}
	\begin{tabular}{|c|c|}
		\hline
		\textbf{Буква} & \textbf{Кодовое слово} \\ \hline
		$A$        & 00                     \\ \hline
		$B$        &                        \\ \hline
		$C$        & 010                    \\ \hline
		$D$        & 011                    \\ \hline
		$E$        & 1011                   \\ \hline
	\end{tabular}
	\quad
	\begin{tabular}{|c|c|}
		\hline
		\textbf{Буква} & \textbf{Кодовое слово} \\ \hline
		$F$        & 1001                   \\ \hline
		$S$        & 1100                   \\ \hline
		$X$        & 1010                   \\ \hline
		$Y$        & 1101                   \\ \hline
		$Z$        & 111                    \\ \hline
	\end{tabular}
\end{center}

Укажите кратчайшее кодовое слово для буквы $B$, при котором код
удовлетворяет условию Фано. Если таких кодов несколько, укажите код с
\textbf{наименьшим} числовым значением.

\textit{Примечание}. Условие Фано означает, что никакое кодовое слово
не является началом другого кодового слова. Это обеспечивает
возможность однозначной расшифровки закодированных сообщений.

\end{document}
