\documentclass[14pt,a4paper]{article}
\usepackage[utf8]{inputenc}
\usepackage[russianb]{babel}
\usepackage[left=1.5cm,right=1.5cm,top=2cm,bottom=2.5cm]{geometry}
\usepackage{setspace}
\usepackage{indentfirst}
\usepackage{amssymb}
\usepackage{amsmath}
\usepackage{bm}

\usepackage{array}
\usepackage[pdftex]{graphicx}
\usepackage{comment}
\usepackage[table,xcdraw]{xcolor}


\usepackage{verbatim}


\graphicspath{{images/}}
\renewcommand{\baselinestretch}{1.3}

\begin{document}

Для начала нужно открыть навигатор. Он находится во вкладе <<Вид>>. Раскрываем третью часть романа, выбираем XII главу, копируем и вставляем в новый документ. Так же копируем и XIV главу. Чтобы посчитать, сколько раз встречается сочетание букв <<по>> или <<По>> в составе других слов, но не как отдельное слово, нужно сначала найти все <<по>>, а потом вычесть из них <<по>> как слова целиком.

Всего <<по>> встречается 112 раз. В качестве слова целиком -- 10 раз. Однако, если посмотреть найденные слова целиком, мы увидим <<по-гречески>>, что по условию -- в составе слова. Поэтому <<по>> как отдельное слово встречается только 9 раз. Получаем ответ: $112 - 9 = 103$

\end{document}
