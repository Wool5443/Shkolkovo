\documentclass[14pt,a4paper]{article}
\usepackage[utf8]{inputenc}
\usepackage[russianb]{babel}
\usepackage[left=1.5cm,right=1.5cm,top=2cm,bottom=2.5cm]{geometry}
\usepackage{setspace}
\usepackage{indentfirst}
\usepackage{amssymb}
\usepackage{amsmath}

\usepackage{array}
\usepackage[pdftex]{graphicx}
\usepackage{comment}
\usepackage[table,xcdraw]{xcolor}


\usepackage{verbatim}


\graphicspath{{images/}}
\renewcommand{\baselinestretch}{1.3}

\begin{document}

Рассматривая пары, нужно проверить, что хотя бы одно число
оканчивается на 2 в пятеричной системе счисления. Для этого можно
посмотреть остаток от деления на 5 чисел из пары -- он нам и даст
последнюю цифру записи.

\begin{verbatim}
# Откроем файл и считаем числа в список a

file = open("17__1z92y.txt")
numbers = [int(i) for i in file]

# Создадим счётчик и переменную для максимальной суммы
count = 0
max_sum = 0

# Перебираем пары соседних элементов
for i in range(len(numbers) - 1):
    # Проверяем, что хотя бы одно число из пары оканчивается на 2 в 5 сс
    if (numbers[i] % 5 == 2) or (numbers[i + 1] % 5 == 2):
        # Увеличиваем счётчик, если условие прошло
        count += 1
        # Обновляем максимальное число
        if numbers[i] + numbers[i + 1] > max_sum:
            max_sum = numbers[i] + numbers[i + 1]

# Выводим ответ
print(count, max_sum)
\end{verbatim}

\end{document}
