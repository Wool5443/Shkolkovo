\documentclass[14pt,a4paper]{article}
\usepackage[utf8]{inputenc}
\usepackage[russianb]{babel}
\usepackage[left=1.5cm,right=1.5cm,top=2cm,bottom=2.5cm]{geometry}
\usepackage{setspace}
\usepackage{indentfirst}
\usepackage{amssymb}
\usepackage{amsmath}
\usepackage{bm}

\usepackage{array}
\usepackage[pdftex]{graphicx}
\usepackage{comment}
\usepackage[table,xcdraw]{xcolor}


\usepackage{verbatim}


\graphicspath{{images/}}
\renewcommand{\baselinestretch}{1.3}

\begin{document}

Для решения задачи нам нужно будет поставить фильтры по дате, району,
типу операции и наименованию товара.

Чтобы узнать район, в ячейку H2 запишем следующую формулу и растянем
вниз:
\begin{center}
    =ВПР(C2;\$Магазин.A:C;2;0)
\end{center}

Для наименования товара в ячейку I2 запишем формулу и растянем:
\begin{center}
    =ВПР(D2;\$Товар.A:F;3;0)
\end{center}

Также нам понадобится вес товара, для этого в ячейке J2 воспользуемся такой
формулой:
\begin{center}
    =ВПР(D2;\$Товар.A:F;5;0)
\end{center}

Теперь посчитаем суммарный вес операции. В ячейку K2 запишем
\begin{center}
    =J2*F2
\end{center}

Теперь поставим фильтры по дате и району, тип операции -- \textit{поступление}, товар -- \textit{крахмал картофельный}. Скопируем полученную таблицу на новый лист и посчитаем ответ, записав в ячейку L2:
\begin{center}
    =СУММ(K:K)
\end{center}

Получим ответ 355.

\end{document}
