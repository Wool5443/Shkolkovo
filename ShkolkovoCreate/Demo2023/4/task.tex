\documentclass[14pt,a4paper]{article}
\usepackage[utf8]{inputenc}
\usepackage[russianb]{babel}
\usepackage[left=1.5cm,right=1.5cm,top=2cm,bottom=2.5cm]{geometry}
\usepackage{setspace}
\usepackage{indentfirst}
\usepackage{amssymb}
\usepackage{amsmath}
\usepackage{bm}

\usepackage{array}
\usepackage[pdftex]{graphicx}
\usepackage{comment}
\usepackage[table,xcdraw]{xcolor}


\usepackage{verbatim}


\graphicspath{{images/}}
\renewcommand{\baselinestretch}{1.3}

\begin{document}

По каналу связи передаются сообщения, содержащие только буквы из
набора: А, З, К, Н, Ч. Для передачи используется двоичный код,
удовлетворяющий прямому условию Фано, согласно которому никакое
кодовое слово не является началом другого кодового слова. Это условие
обеспечивает возможность однозначной расшифровки закодированных
сообщений. Кодовые слова для некоторых букв известны: Н – 1111, З –
110. Для трёх оставшихся букв А, К и Ч кодовые слова неизвестны.
Какое количество двоичных знаков потребуется для кодирования слова
КАЗАЧКА, если известно, что оно закодировано \textbf{минимально}
возможным количеством двоичных знаков?

\end{document}
